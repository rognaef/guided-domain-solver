% ==================================================================================
% Satzspiegel, Zeichensatz und Schriftgr�ssen
% ==================================================================================
\newcommand{\myfonttype}{ptm}     								% ptm = Times, phv = Helvetica, pcr = Courier
\usepackage[onehalfspacing]{setspace}					    % Zeilenabstände einstellen auf standardmässig 1,5fachen Zeilenabstand
\usepackage[greek,ngerman]{babel}	 								% Erscheinungsbild auf Deutsch umstellen
\usepackage[utf8]{inputenc}											  % Umlaute zulassen
\usepackage[babel,german=quotes]{csquotes}				% Quotes auf Deutsch umstellen
\usepackage[T1]{fontenc}													% Verbesserte Silbentrennung									
\usepackage{lmodern}															% Darstellung der Schrift im PDF verbessern
\usepackage{microtype}														% Verbessert Randausgleich bei langen W�rtern
\usepackage{textgreek}
\usepackage{upgreek}
\usepackage{lscape}

\setlength{\parindent}{0cm}                       % Einrückung bei Beginn eines Absatzes
\setlength{\parskip}\medskipamount								% Abstand zwischen zwei Abs�tzen

%\renewcommand*{\chapterheadstartvskip}{\vspace*{8mm}} % Abstand zwischen Kopfzeile und Kapitelüberschrift

% ==================================================================================
% Kopf- und Fusszeile
% ==================================================================================
\usepackage[headsepline]{scrlayer-scrpage} 		   % Kopf/Fusszeile manuell definieren
\automark[section]{section}             				 % Kapitel auf [linker] und {rechter} Seite
\ihead{\headmark}                                % Kopfzeilentext Innen
\chead{}														             % Kopfzeilentext Mitte
\ohead{}                                         % Kopfzeilentext Aussen
\pagestyle{scrheadings}							             % Kopf/Fusszeile erstellen
\setkomafont{pageheadfoot}{\normalfont}

% ==================================================================================
% Schusterjungen und Hurenkinder
% ==================================================================================
\clubpenalty  = 10000 				                   % Schusterjungen verbieten
\widowpenalty = 10000 				                   % Hurenkinder verbieten
\displaywidowpenalty = 10000                     % Hurenkinder vor abgesetzter math. Formel

% ==================================================================================
% Gleitobjekte (Tabellen und Abbildungen)
% ==================================================================================
\usepackage{tabularx} 							 % Erweiterte Tabellenumgebung
\usepackage{multirow}								 % Zeilen in Tabellen können verbunden werden
\usepackage{booktabs}                % Erm�glicht \toprule, \midrule und \bottomrule
\usepackage{subfig}
\usepackage{graphicx}	  		 				 % Einbinden von Bildern ermöglichen
\usepackage[section]{placeins}       % Bilder sind immer im zugehörigen Kapitel
\usepackage{wrapfig}								 % von Schrift umflossene Grafiken
\usepackage{array}
\usepackage{longtable}
%\captionsetup[subfigure]{position=top,singlelinecheck=off,justification=raggedright}

% ==================================================================================
% Mathematische Formeln
% ==================================================================================
\usepackage{amsmath,amssymb,amsbsy}        % Mathematische Symbole
\usepackage[version=4,arrows=pgf]{mhchem}  % Chemische Formeln
\usepackage{xfrac}									       % Ermöglicht das korektes Setzen von Br�chen im  
  																	       % Fliesstext mit dem Befhel "sfrac"														
\usepackage{units}												 % Korrektes Setzen von Einheiten im Fliesstext

\usepackage{pdfrender}
\newcommand*{\bg}[1]{%
  \textpdfrender{%
    TextRenderingMode=FillStroke,%
    LineWidth=.25pt,%
  }{#1}%
}

% ==================================================================================
% PDF
% ==================================================================================
% Darstellung und Verlinkungen im pdf-Dokument einstellen
\usepackage[hidelinks,								% Links als normaler Text darstellen
	pdfpagemode = UseNone,							% Lesezeichen im pdf-Reader nicht anzeigen
	pdfpagelayout = TwoColumnRight,			% Seitenanzeige des pdf-Dokuments angeben
	pdfauthor = {OST- ICE},		          % Autor des pdf-Dokuments
	pdftitle = {Bachelorarbeit}]  			% Titel des pdf-Dokuments
	{hyperref}

% ==================================================================================
% Bibliographie
% ==================================================================================
\usepackage[backend=biber,style=alphabetic-verb,sorting=anyt,firstinits=true,
            minbibnames=3,maxbibnames=3,isbn=false,url=false,doi=false,eprint=false,
						singletitle=true,bibwarn=true,hyperref=true]{biblatex}	
							
\addbibresource{./bibliography/bibliography.bib}                           % Quelle f�r Bibliographie
\AtEveryBibitem{																													 % Einträge aus Bibtex-Datei l�schen
	\clearfield{number}              																				 % Entfernt Nummer/Issue
	\clearfield{issue}
	\clearfield{note}                                                        % Entfernt Notizen
	\clearfield{abstract}                                                    % Entfernt Abstract
	\clearfield{pagetotal}
	\clearlist{language}
}

\setlength\bibitemsep{2mm}																								 % Abstand im zwischen Quellen
\DeclareNameAlias{default}{family-given}																		 % erst Nachname, dann Vorname
\DeclareFieldFormat*[article]{journaltitle}{#1}  					 								 % Kursiv und "..." entfernen
\DeclareFieldFormat*[article,book,incollection]{title}{#1}  				       % Kursiv und "..." entfernen
\DeclareFieldFormat*[incollection]{booktitle}{#1}				  								 % Kursiv und "..." entfernen
\renewcommand*{\finalnamedelim}{\addsemicolon\space}											 % Ersetzt "und" vor letztem Namen durch Semikolon
\renewcommand*{\multinamedelim}{\addsemicolon\space}                       % Semikolon als Trenner zwischen den Namen
\renewcommand*{\labelnamepunct}{\addcolon\space}													 % Doppelpunkt nach dem letzten Namen
\renewcommand*{\labelalphaothers}{}										                     % Kein + bei mehreren Autoren
\renewcommand*{\multicitedelim}{\addcomma\space}			                     % Komma als Seperator bei mehreren Zitaten
\renewbibmacro*{in:}{\ifentrytype{article}{}{\printtext{\bibstring{In}\intitlepunct}}}  % Kein "In" bei Artikel
\DefineBibliographyStrings{german}{andothers = {{et\,al\adddot}},}         % et al. anstatt u.a.
\patchcmd{\bibsetup}{\interlinepenalty=5000}{\interlinepenalty=10000}{}{}  % Seitenumbruch innerhalb Quelle vermeiden

\DeclareLabelalphaTemplate{ 
  \labelelement{ 
    \field[uppercase, final]{shorthand} 
    \field[uppercase, final]{label} 
    \field[uppercase,strwidth=3,strside=left,names=1]{labelname}           % nur die ersten drei Buchstaben des ersten Autors
		\field[uppercase,strwidth=3,strside=left,names=1]{journaltitle}        % nur die ersten drei Buchstaben der Zeitschrift
		\field[uppercase,strwidth=3,strside=left,names=1]{title}               % nur die ersten drei Buchstaben des Titels
   } 
  \labelelement{ 
    \field[strwidth=2,strside=right]{year}                                 % die letzten beiden Buchstaben des Jahres 
  } 
} 

\renewbibmacro*{journal+issuetitle}{% 
   \usebibmacro{journal}% 
   \setunit*{\adddot\addspace}%<--da 
   \iffieldundef{series} 
   {} 
   {\newunit 
      \printfield{series}% 
      \setunit{\addspace}}% 
   \usebibmacro{volume+number+eid}% 
   \setunit{\addspace}% 
   \usebibmacro{issue+date}% 
   \setunit{\addcolon\space}% 
   \usebibmacro{issue}% 
   \newunit} 
   
\usepackage{float}
\usepackage{comment}